\documentclass[margin,line]{resume}
 
\usepackage[latin1]{inputenc}
\usepackage[english,french]{babel}
\usepackage[T1]{fontenc}
\usepackage{fontawesome}
\usepackage{graphicx,wrapfig}
\usepackage{url}
\usepackage{mathrsfs}
\usepackage{enumitem}
\usepackage{dirtytalk}
\usepackage[colorlinks=true, pdfstartview=FitV, linkcolor=blue, citecolor=blue, urlcolor=blue]{hyperref}
\pdfcompresslevel=9


\begin{document}{\sc \Large \textbf{Som Dev Bishoyi}}
\begin{resume}

% === PICTURE ===

    % \vspace{0.5cm}
    % \begin{wrapfigure}{R}{0.15\textwidth}
    %      \vspace{-0.9cm}
    %     \begin{center}
    %     \includegraphics[width=0.15\textwidth]{reveyrand_face.png}
    %     \end{center}
    %      \vspace{-1cm}
    % \end{wrapfigure}

% === PERSONAL INFO ===
 
    \section{\mysidestyle Contact Information}
    University of Massachusetts Dartmouth \hspace{2cm} +1 (508)8581220 \\
    Department of Mathematics \hspace{4cm} \href{https://sdbishoyi.github.io/}{sdbishoyi.github.io }\\ 
    64 Thomas street, New Bedford, MA 02747 \hspace{1.5cm}
    \faGithub \space \large{\href{https://github.com/sdbishoyi}{https://github.com/sdbishoyi}}\\
    % \faTv  \space \href{http://www.microwave.fr}{www.microwave.fr} \\ 
    % \faTwitter  \space \href{https://twitter.com/reveyrand/}{@reveyrand} \\
    % \faLinkedin \space \href{http://www.linkedin.com/in/reveyrand/}{www.linkedin.com/in/reveyrand} \\ 
    % \faYoutubePlay  \space \href{https://www.youtube.com/c/tibaultreveyrand/}{www.youtube.com/c/tibaultreveyrand}
   
% === OBJECTIVE ===

    % 1st attempt
    \section{\mysidestyle Education}
    {\large\textbf{University of Massachusetts Dartmouth}}\\
    % \hrulefill\\
    [0.5ex]
    \hfill{\large Ph.D. in Computational Science and Engineering,} \hfill \textit{Sept 2022-present}\\
     {\large Specialization in Numerical PDEs, Scientific ML,\\
     Gravitational waves, GPA: 4.00/4.00}\\
    % \hspace{5mm} 
    % \hrulefill\\[-2ex]
    [0.5ex]
    {\large Dissertation Advisor: Professor Scott Field}

% \end{itemize}
    {\large\textbf{IISER Kolkata}}\\
    [0.5ex]
       {\large BS-MS in Physics, Minor in Mathematics,} \hfill \textit{August 2017-July 2022}\\ {\large GPA: 9.05/10} \\
       [0.5ex]
       {\large Thesis: \href{http://eprints.iiserkol.ac.in/1378/}{Studies on the spacetime of slowly rotating stars}}\\
      
      % ==== RESEARCH INTERESTS ===
      \section{\mysidestyle Research Interests}
        {\large I develop efficient, high-order PDE solving algorithms for problems in computational general relativity, black hole perturbation theory(BHPT) and gravitational waves. I work with discontinuous Galerkin/WENO methods to solve wave equations in BHPT. Also developing exact nonreflecting boundary conditions using Boundary Kernels and perfectly matched layers/hyperboloidal slicing.}
        
       % {\large General relativity, black hole perturbation theory, gravitational waves, extreme mass ratio inspiral simulations, discontinuous Galerkin/WENO methods. Specifically, using spectrally accurate methods to construct simulations of EMRIs. 

       
       % Also interested in mathematical general relativity, Aretakis extremal instability, no hair theorem, price's law and decay rates of perturbations in extremal black holes.
       % }
       % === PUBLICATION SUMMARY ===
     \section{\mysidestyle Publications \& preprints}\vspace{0mm}
      % \begin{description}
      %     \item [\large {Som Dev Bishoyi, Subir Sabharwal, Gaurav Khanna, "Non-Axisymmetric Gravitational "Hair" \\
      %     for Extremal Black Holes", Submitted to GRG, \href{https://arxiv.org/abs/2407.06926}{https://arxiv.org/abs/2407.06926} }	]	
      % \end{description}
      % \begin{list}
      %     \item {\large {Som Dev Bishoyi, Subir Sabharwal, Gaurav Khanna, "Non-Axisymmetric Gravitational "Hair" for Extremal Black Holes", Submitted to GRG, \href{https://arxiv.org/abs/2407.06926}{https://arxiv.org/abs/2407.06926} } }
      % \end{list}
         \large {1. Som Dev Bishoyi, Subir Sabharwal, Gaurav Khanna, \textit{Numerical Evidence for Non-Axisymmetric Gravitational \say{Hair} for Extremal Kerr Black Hole Spacetimes with Hyperboloidal Foliations}}, \href{https://www.doi.org/10.1007/s10714-025-03378-1}{10.1007/s10714-025-03378-1}. \\
         \\
         \large {2. Som Dev Bishoyi, Subir Sabharwal, Gaurav Khanna, \textit{Source-Driven Tails in Kerr Spacetime: Nonlinear effects in Late-Time Behavior}, appears on arXiv on Jan 20}\\
        \\
        \large {3. Som Dev Bishoyi, Scott Field, Stephen Lau, \textit{Exact radiation outer boundary conditions and near-to-far field signal teleportation for the Bardeen-Press-Teukolsky equation}, in preparation}.
    % === SKILLS ===
    % \section{\mysidestyle Skills}\vspace{2mm}       
    % \begin{description}
   	% 		\item[Operating systems:] DOS, Windows, Unix and Linux.
				% \item[Programming languages:] Pascal, 80x86 Assembler, C, C++, TCL/TK, JAVA, PHP, mySQL.
				% \item[Office softwares:] Microsoft Office, Open Office, LaTeX, DocBook.
				% \item[System Level Modeling:] Amplifiers, modulators and mixers with splines, neural networks or Volterra expansions. Bilateral Modeling by PhD model.
				% \item[Circuit Level Modeling:] Linear, nonlinear and electrothermal models of HEMTs.
				% \item[Languages:] French, English.
    % \end{description}
     \vspace{-2ex}
    %=== RESEARCH EXPERIENCE
    \section{\mysidestyle Research Experience}  	            
       \begin{description}        
     %   		\item[Measurement Engineer (CNRS)]\small{XLIM \hfill \textsl{June 2016-Present}}\\ 	   
     %        \item[Lecturer]\small{University of Colorado, Boulder \hfill \textsl{January 2016-May 2016}}\\
    	% 			ECEN 5014-003, ``Microwave Measurements and Calibration Fundamentals''
  			% \vspace{2mm}

        \item[ROBCs and near-to-far field teleportation for the BPT equation] \hfill
            \vspace{1ex}
        % Achievements:
            \begin{list2}
                    % \item{Solving massless scalar wave equation in RN and Kerr spacetime}
                    \item {Developed boundary kernels to construct exact non-reflecting BCs to numerically solve the Bardeen-Press-Teukolsky equation in Schwarzschild spacetime.}
                    \item{Developed teleportation kernels that perform near to far field teleportation of signals and extract them in the astrophysical wave zone.}
                    \item {The above techniques resulted in long duration stable solutions of the BPT equation in Boyer-Lindquist coordinates, an open problem earlier.}
                    % \item{}
                \end{list2} 

                        
   		\item[General spin-weight time domain Teukolsky equation solver] \hfill
        % Achievements:
        \vspace{1ex}
            \begin{list2}
                    \item{Discontinuous Galerkin(DG) solver for gravitational perturbations in the time domain having spectral convergence.}
                    \item{Computing the Teukolsky source term for gravitational perturbations.}
                    \item{Formulating a symmetric hyperbolic system of coupled PDEs using different choices of auxilliary variables and hyperboloidal coordinates.}
                    \item {Implemented hyperboloidal slices (a variant of the perfectly matched layers technique) in the DG method to evolve wave equations (Bardeen-Press-Teukolksy equation).}
                    % \item{}
                \end{list2}   	
        \vspace{2mm}

%         \item[Extremality in Reissner Nordstrom(RN) and Kerr BHs ] \hfill
%             \vspace{1ex}
%         % Achievements:
%             \begin{list2}
%                     % \item{Solving massless scalar wave equation in RN and Kerr spacetime}
%                     \item {Used Discontinuous Galerkin method to numerically solve the massless scalar wave                             equation in RN and extremal RN spacetime.}
%                     % \item{Using hyperboloidal slices to get late time tails at fixed $r_{*}$ and at                                   $\mathscr{I}^{+}$}
%                      \item {Implemented hyperboloidal slices in the DG method for calculating local tail                                 decay rates for static and generic initial      data at some finite distance and at                                  $\mathscr{I^+}$ for RN and extremal RN black holes.}
%                     % \item{}
%                 \end{list2}   
%                 \vspace{2mm}
        
 				
 			% \item[Research Associate - Visiting Scholar]\small{University of Colorado at Boulder \hfill \textsl{February 2012-July 2012}}\\
 			% 	    GaN HEMTs based rectifiers characterizations and analysis
 			\vspace{2mm}

        \end{description}
        \vspace{-0.5cm}    

        \newpage
                            % === HONORS/AWARDS ===
     %%%%%%%%%%%%%%%%%%%%%%%%%%%%%%%%%%%%%%%%%%%%
      \section{\mysidestyle Awards}\vspace{4mm} %
     %%%%%%%%%%%%%%%%%%%%%%%%%%%%%%%%%%%%%%%%%%%%
      \begin{list2}
        \item \textbf{UMass Dartmouth Doctoral Fellowship}, Research fellowship for a period of 1 year. \hfill 
        \textit{Sept 2022 to May 2023}
        \vspace{1ex}
        \item \textbf{UMass Dartmouth Provost Travel Grant}, Travel grant of \$500 for presenting at conferences. \hfill 
        \textit{APS April Meetings 2024, 2025, 2026}
        \vspace{1ex}
        \item \textbf{APS DGRAV Travel Grant},  Travel grant of \$300 for presenting at APS april meeting 2024. \hfill \textit{April 3 2024 - April 6 2024}
        \vspace{1ex}
        \item \textbf{IISER-K Summer Fellowship} Fellowship of 10,000 rupees for research project on Ahoronov-Bohm effect and geometric phases. \hfill \textit{May 2019 - July 2019}
        \vspace{1ex}
        \item \textbf{IIT Indore Research Internship} Internship for two months on
        cosmological N Body Simulations . \hfill \textit{May 2020 - July 2020}
        \vspace{1ex}
%         \textsl{\footnotesize{T. Reveyrand, C. Maziere, J.M. N\'ebus, R. Qu\'er\'e, A. Mallet, L. Lapierre, J. Sombrin, ``A calibrated time domain envelope measurement system for the behavioral modeling of power amplifiers'', European Microwave Week, GAAS 2002, pp. 237-240, Milano, September 2002}}
%         \vspace{1mm}
%         \item \textbf{Best Student Paper Award}, Journ\'ees Nationales Micro-ondes (JNM), 2007 \\
% \textsl{\footnotesize{
% O. Jardel, F. De Groote, T. Reveyrand, C. Charbonniaud, J.P. Teyssier, R. Qu\'er\'e, D. Floriot, ``Mod\'elisation du drain-lag dans les mod\`eles \'electriques grand-signaux de transistors HEMTs AlGaN/GaN'', 15eme Journ\'ees Nationales Micro-ondes (JNM),3C1, Toulouse, Mai 2007.}}
      \end{list2}
        \vspace{-2ex}	
%     	Up to 130 other refrences are available here : \\ 
%     	\vspace{-8mm}

% \href{http://www.microwave.fr/publications.html}{http://www.microwave.fr/publications.html}
 	%\vspace{-1mm}
 	 % \newpage

           
             %****************************************************
    	\section{\mysidestyle Invited Talks}\vspace{2mm}                                  %****************************************************
  	      1. Scalar and gravitational horizon hair as observable imprints of extremal black holes, \\ \hfill \textit{Extremal Black Holes and Black hole thermodynamics workshop, ICERM, Brown University, Jan-2026}
 	              % \\
                % 2.   
 	

     
                  % === CONTRIBUTED TALKS
        %****************************************************
  	\section{\mysidestyle Contributed Talks and posters}\vspace{2mm}                                 %****************************************************
                1. Exact boundary conditions for the Teukolsky equation, \textit{Prospects in Theoretical Physics workshop, Institute of Advanced Study, Princeton, July 2025} \\
                2. Exact boundary conditions for the Teukolsky equation, \textit{SciML for gravitational wave astronomy workshop, ICERM, Brown University, June 2025}\\
                3. Long simulations of extreme mass ratio inspirals by solving the Teukolsky equation with singular source terms, \textit{APS April meeting 2025} \\ 
  	      4. Determining extremality of Reissner-Nordstrom BHs using late time tails at $\mathscr{I^{+}}$, 
          \\ \hfill \textit{APS April meeting 2024}
%  	      \begin{list2}
%  	      \item{``Microwave Theory and Techniques'' society  \hfill \textsl{2007-present}}
%  	      \item{``Instrumentation and Measurement'' society  \hfill \textsl{2007-present}}
%  	      \item{MTT-11 ``Microwave Measurements'' technical committee \hfill \textsl{2009-present}}
%  	      \item{IEEE MTT-S Technical Program Review Committee (TPRC) for IMS  \hfill \textsl{2013-present}}
%  	      \item{Judge for IEEE MTT-S Graduate Fellowships  \hfill \textsl{2014-present}}
%  	      \item{Chair for IEEE Denver Section Jt. Chapter, AP03/MTT17  \hfill \textsl{2015-2016}}
 	      
%  	      \end{list2}
%  	   \textbf{The European Microwave Association (EuMA)}\hfill \textsl{2009-2015}
 	
    
% === HISTORY ===
           
 	    \section{\mysidestyle Teaching Experience}
                \begin{description}
   		    \item[Department of Mathematics, UMass Dartmouth]\hfill%
            \vspace{2mm}
                            \begin{list2}
  	      		 	   	\item{TA for High Performance Scientific Computing \hfill \textit{Sept 2024 to Dec 2024}}
 	      				% \item{}
 	      				% \item{}
 	      				% \item{}
 	      			\end{list2}
            % \small{University of Colorado at Boulder \hfill \textsl{June 2013-May 2016}}\\ Achievements:
  	   		   	
 	      	\vspace{2mm}
 
   
    		\item[Department of Physical Sciences, IISER Kolkata]\hfill%
            \vspace{2mm}
  	      		\begin{list2}
 	      				\item{TA for Classical Mechanics II \hfill \textit{August 2020 to Dec 2020}}
 	      				\item{TA for Classical Mechanics II \hfill \textit{August 2021 to Dec 2021}}
 	      				\item{TA for Introductory Electromagnetism \hfill \textit{January 2021 to May 2021}}
 	      				% \item{}
 	      			\end{list2}   
 			\vspace{2mm}
 			
 				
 			\item[Computation and Data Sciences, IISER Kolkata] \hfill%
            \vspace{2mm}%\vspace{2mm}
 			% 
            \begin{list2}
                \item{TA for Scientific Computing in Python \hfill \textit{January 2022 to May 2022}}             
                \end{list2}
 				
   %   		\item[Research Engineer (CNRS)]\small{XLIM \hfill \textsl{May 2005-November 2007}}\\
   % 			Achievements:
  	%         \begin{list2}
 	 %      		\item{Frequency domain load-pull measurement setup (VNA in receiver mode with pulse capabilities) developpemnt with Scilab (calibration procedures, measurement automation, data processing)}
 	 %      		\item{Large signal caracterization of transistor (mainly european GaN in the framework of Korrigan}
 	 %      		\item{Korrigan WP3.3 workpackage leader in Korrigan. Developpement of a internet database (Php / mySQL) to let partners share data and informations}
 	 %      		\item{GaN HEMTs ``spice-like'' nonlinear models}
 	 %      	\end{list2}  	      
 		% 	\vspace{2mm}
     		
            
   %          \item[Research Engineer]\small{NMDG Engineering bvba \hfill \textsl{November 2004-February 2005}}\\
   %  			Implementation of the High Impedance Probe module (calibration and measurements) in the commercial LSNA Software (based on Mathematica)
 
			% \vspace{2mm}
     		
            
   %          \item[Postdoctoral scientist]\small{CNES (French Space Agency) \hfill \textsl{October 2003-September 2004}}\\
   %  			Development of characterization tools interfaces within the free open-source scientific package Scilab
 
   % 			\vspace{2mm}
     		
            
   %          \item[Postdoctoral scientist]\small{CNES (French Space Agency) \hfill \textsl{October 2002-September 2003}}\\ 	     	
 		% 			Achievements:
  	%       	\begin{list2}
 	 %      		\item{Large Signal Network Analysis (LSNA) characterizations in time-domain}
 	 %      		\item{Development of a new LSNA module in order to investigate time domain waveforms at internal nodes of MMICs with high impedance probes (HIP) to validate circuits designs and to analyze nonlinear parametric stability}
 	 %      		\item{Large Signal Network Analysis (LSNA) characterizations in time-domain}
 	 %      	\end{list2}  	      
 		% 	\vspace{2mm}
     		
            
   %          \item[Researcher]\small{IRCOM / University of Limoges \hfill \textsl{October 1998-September 2002}}\\ 	     	
 		% 			Achievements:
  	%       	\begin{list2}
 	 %      		\item{Development of the RF time-domain envelope measurement setup (hardware and software)}
 	 %      		\item{Development of the calibration procedure of the time-domain envelope measurement setup}
 	 %      		\item{Power amplifiers characterizations : Load-pull, IM3, NPR}
 	 %      		\item{Behavioral modeling of nonlinear devices with memory effects for system level}
 	 %      		\item{Development of a dynamic complex gain model with neural networks}
 	 %      	\end{list2}  	      
  	% 		\vspace{2mm}
     		
            
   %          \item[Lecturer]\small{University of Limoges \hfill \textsl{October 1998-September 2002}}\\
   %  				RF devices, analog/digital communication systems, signal processing, propagation waves...
			% \vspace{2mm}
     		
            
   %          \item[Postgraduate student]\small{IRCOM / University of Limoges \hfill \textsl{February 1998-July 1998}}\\
   %  				Circuits level simulations of IM3 and NPR in order to optimize the trade-off between linearity and efficiency
    				
             \end{description}  
    
 	  \section{\mysidestyle Additional Expertise} 
 	   \begin{description}
 	       % \item[languages]
           \item [Computing]:  C, Python(scipy, numpy, sympy, JAX), numerical PDE solvers,
           pseudo-spectral methods, finite difference methods, Matplotlib, Mathematica
            xAct, LaTeX, git
            \vspace{2ex}
            \item [Organizing]:  Biweekly talks at meetings of gravity research groups of UMass Dartmouth and University of Rhode Island.
            \vspace{2ex}
            \item[Research Mentorship]:
            \item Project: \textbf{WENO solver for the massive Klein-Gordon equation} 
            \vspace{1ex}
            \begin{list2}
                \item{Varenya Upadhaya, 2nd year PhD student at UMass Dartmouth, Spring 2025}
            \end{list2}
            \item Project: \textbf{DG solver for wave equations in modified gravity spacetimes}
            \vspace{1ex}
            \begin{list2}
                \item{Scott Shaw, MS-Physics student at UMass Dartmouth, Spring 2023}
                \item{Ansh Gupta, MS-Physics student at IISER Mohali, 2023}
            \end{list2}    
 	   \end{description}	   
 	       
\end{resume}   
\end{document}